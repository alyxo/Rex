\documentclass[a4paper]{report}
\usepackage[utf8]{inputenc}  
\usepackage[T1]{fontenc}  

\renewcommand{\chaptername}{}

\usepackage{helvet}
\renewcommand{\familydefault}{\sfdefault}

\usepackage[a4paper]{geometry}
\geometry{hscale=0.85,vscale=0.85,centering}

\begin{document}
\chapter{Introduction}
\section{Objet du document}

Ce document est le Manuel Utilisateur de l’application informatique GESICO destinée aux clients de SNCF Réseau. GESICO est l’application de gestion des sillons commandés. Ce manuel décrit les fonctionnalités disponibles pour gérer et suivre les commandes de sillons.

\section{Terminologie et Abréviations}	
\begin{table}[h]
	\begin{tabular}{|l|l|}
		\hline
		CH & code \textbf{CH}antier \\
		\hline
		CI & \textbf{C}ode \textbf{I}nfrastructure \\
		\hline
		DS & \textbf{D}emande au \textbf{S}ervice \\
		\hline
		DSA & \textbf{D}emande au \textbf{S}ervice \textbf{A}daptation 
		\\
		\hline
		DRR & \textbf{D}ocument\textbf{ }de\textbf{ R}éférence du 
		\textbf{R}éseau \\
		\hline
		DTS & \textbf{D}emande \textbf{T}ardive au \textbf{S}ervice \\
		\hline
		DTSF & \textbf{D}emande Tardive au Service au Fil de l'eau \\
		\hline
		EF & \textbf{E}ntreprise \textbf{F}erroviaire \\
		\hline
		GESICO & \textbf{GE}stion des \textbf{SI}llons \textbf{CO}
		mmandés \\
		\hline
		GI & \textbf{G}estionnaire d'\textbf{I}nfrastructure \\
		\hline
		GID & \textbf{G}estionnaire d'\textbf{I}nfrastructure \textbf{D
		}élégué \\
		\hline
		HOUAT & \textbf{H}oraires \textbf{U}tiles \textbf{A} \textbf{T
		}ous \\
		\hline
		SDM & \textbf{S}illon de \textbf{D}erniere \textbf{M}inute \\
		\hline
		TCT & \textbf{T}ype de \textbf{C}onvoi \textbf{T}rafic \\
		\hline
		THOR & \textbf{T}racé des \textbf{HOR}aires \\
		\hline
		TTH & \textbf{T}rain \textbf{T}ype \textbf{H}ouat \\
		\hline
		UI & \textbf{U}tilisateur \textbf{I}nfrastructure \\
		\hline
		VDS & \textbf{V}ie \textbf{D}u \textbf{S}illon \\
		\hline
		ECS & \textbf{E}tat \textbf{C}ourant du \textbf{S}illon \\
		\hline
		SJE & \textbf{S}illon \textbf{J}our à l'\textbf{E}tude \\
		\hline
	\end{tabular}
\end{table}

\section{Conformité des navigateurs WEB}

\begin{itemize}
	\item Navigateur : Internet Explorer 8 ou Firefox 3.5.7 \\
	\textit{L'application est développée pour ces 2 types de navigateurs et pour les 
	versions indiquées (en version 32bits).}
	\item OS : Windows XP \\
	\textit{L’application est développée pour Windows XP.}
	\item Acrobat Reader 7.0 ou supérieur.
\end{itemize}

\end{document}